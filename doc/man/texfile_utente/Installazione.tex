\section{Installazione e Configurazione}
\label{cap:installazione}
\subsection{Requisiti di sistema}
\begin{itemize}
    \item Sistema operativo: Windows 10+ / macOS 10.14+ / Linux
    \item Processore: almeno dual-core 2.0 GHz
    \item Memoria RAM: minimo 2 GB (8 GB raccomandati)
    \item Spazio su disco: minimo 200 MB per l'installazione
    \item Java Runtime Environment 11+ installato e configurato nel \texttt{PATH}
    \item PostgreSQL 13+ con istanza attiva sulla porta \texttt{5432} (per il server)
    \item Libreria \href{https://gluonhq.com/products/javafx/}{JavaFX 17.0.15} installate (per il client)
\end{itemize}

\subsection{Installazione e avvio del server}
\begin{enumerate}
    \item Ottenere il pacchetto \texttt{Server.jar} più recente 
    dalla sezione \emph{Release} della repository ufficiale: \href{https://github.com/ozneroL541/TheKnife}{https://github.com/ozneroL541/TheKnife}.
    \item Creare un database PostgreSQL:
    \begin{minted}{SQL}
    CREATE DATABASE theknife;
    CREATE USER theknife WITH ENCRYPTED PASSWORD 'password';
    GRANT ALL PRIVILEGES ON DATABASE theknife TO theknife;
    \end{minted}
    \item Avviare il server:
    \begin{minted}{bash}
    java -jar Server.jar
    \end{minted}
    \item Una volta avviato, il server richiederà le credenziali di accesso al database.
    Le credenziali sono:
    \begin{itemize}
        \item \textbf{Username:} \texttt{theknife}
        \item \textbf{Password:} \texttt{password}
    \end{itemize}
\end{enumerate}

\subsection{Installazione e avvio del client}
\begin{enumerate}
    \item Ottenere il pacchetto \texttt{ClientTK.jar} più recente 
    dalla sezione \emph{Release} della repository ufficiale: \href{https://github.com/ozneroL541/TheKnife}{https://github.com/ozneroL541/TheKnife}.
    \item Avviare il client:
    \begin{minted}{bash}
    java --module-path ~/path/to/javafx-sdk-17.0.15/lib \
    --add-modules javafx.controls,javafx.fxml -jar ClientTK.jar
    \end{minted}
    Sostituire \texttt{~/path/to/javafx-sdk-17.0.15/lib} con il percorso corretto della libreria JavaFX.
    \item Comparirà la finestra di login, pronta per l'autenticazione.
\end{enumerate}
