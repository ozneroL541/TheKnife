\section{Installazione e Avvio}
\label{cap:installazione}
\subsection{Requisiti di sistema}
\begin{itemize}
    \item Sistema operativo: Windows 10+ / macOS 10.14+ / Linux
    \item Processore: almeno dual-core 2.0 GHz
    \item Memoria RAM: minimo 2 GB (8 GB raccomandati)
    \item Spazio su disco: minimo 200 MB per l'installazione
    \item \href{www.java.com}{Java 22+}
    \item Porta 1099 disponibile per il server e raggiungibile dal client
    \item \href{www.postgresql.org}{PostgreSQL} 13+ con istanza attiva sulla porta \texttt{5432} (per il server)
    \item Libreria \href{https://gluonhq.com/products/javafx/}{JavaFX 17.0.15} installata (per il client)
\end{itemize}

\subsection{Ottenimento dei sorgenti}
Scaricare la versione più aggiornata di \textit{TheKnife} da Github.\\
\href{https://github.com/ozneroL541/TheKnife}{https://github.com/ozneroL541/TheKnife}

\subsection{Compilaione del progetto}
\label{sec:compilazione}
Per la compilazione è richiesto \href{maven.apache.org}{Maven} 3.6+ installato e configurato correttamente.\\
\`E richiesto che esista un utente di PostgreSQL con permessi 
di creazione di database e ruoli nominato come \textit{postgres} e con 
password \textit{postgres} per compilare il progetto e popolare correttamente 
il database.\\
In caso il nome o la password di tale utente differiscano da quelli 
indicati
è necessario modificare opportunamente le righe \textbf{69} e \textbf{70} 
del file \textit{TheKnife/server/pom.xml}.
\begin{minted}{xml}
<configuration>
    <!-- Default connection configuration (for admin operations) -->
    <driver>org.postgresql.Driver</driver>
    <url>jdbc:postgresql://localhost:5432/postgres</url>
    <username>postgres</username>
    <password>postgres</password> <!-- Change with your postgre admin password -->
    <autocommit>true</autocommit>
</configuration>
\end{minted}
La creazione del database e la sua popolazione avvengono in fase di 
compilazione del progetto tramite Maven.
Per compilare il progetto, aprire un terminale nella cartella 
\textit{TheKnife} e lanciare i comandi:
\begin{minted}{bash}
mvn clean install
mvn install
mvn clean
\end{minted}
Se la compilazione è avvenuta con successo si avrà:
\begin{itemize}
    \item Un database PostgreSQL chiamato \texttt{theknife\_db} 
    con proprietario un utente chiamato \texttt{theknife}
    e tabelle popolate.
    \item I due eseguibili \texttt{Server.jar} e \texttt{ClientTK.jar} presenti
    nella cartella \textit{TheKnife/bin}.
\end{itemize}

\subsubsection{Troubleshooting}
\paragraph{Errore nella creazione del database}
In caso di errore nella creazione automatizzata del database è 
possibile crearlo manualmente. 
Leggere la sezione \textbf{\ref{sec:creazione_db} Creazione del database} 
per le istruzioni dettagliate sulla creazione del database e le sue tabelle.
\paragraph{Errore generico}
In caso di problemi durante la compilazione, verificare:
\begin{itemize}
    \item Che il database PostgreSQL sia in esecuzione e accessibile.
    \item Che le credenziali dell'utente PostgreSQL siano corrette.
    \item Che Maven sia installato e configurato correttamente.
\end{itemize}

\subsection{Creazione del database}
\label{sec:creazione_db}
\'E consigliato creare il database in maniera automatizzata
tramite \textit{Maven} in fase di compilazione seguendo le 
istruzioni della sezione precedente (\ref{sec:compilazione}).\\
In caso si desideri creare il database manualmente, seguire i 
passi indicati di seguito.
\begin{enumerate}
    \item Creare il database tramite il comando presente 
    nel file 
    \textit{TheKnife/server/src/main/resources/db/create\_database.sql}
    \item Creare le tabelle con gli script presenti nella cartella
    \textit{TheKnife/server/src/main/resources/db/tables/}
    eseguendoli nel seguente ordine:
    \begin{enumerate}
        \item \texttt{addresses.sql}
        \item \texttt{users.sql}
        \item \texttt{restaurants.sql}
        \item \texttt{favorites.sql}
        \item \texttt{reviews.sql}
    \end{enumerate}
    \item Se lo si desidera, popolare le tabelle utilizzando gli script
    presenti nella cartella\\
    \textit{TheKnife/src/main/resources/db/samples/} 
    eseguendoli nel seguente ordine:
    \begin{enumerate}
        \item \texttt{ex\_addresses.sql}
        \item \texttt{ex\_users.sql}
        \item \texttt{ex\_restaurants.sql}
        \item \texttt{ex\_favorites.sql}
        \item \texttt{ex\_reviews.sql}
    \end{enumerate}
\end{enumerate}
una volta eseguiti gli script il database sarà correttamente 
popolato e sarà possibile avviare il server 
di \textit{TheKnife}.

\subsection{Avvio del server}
\label{sec:avvio_server}
Per avviare il server è necessario che la creazione del database 
e delle relative tabelle sia avvenuta con successo 
(\ref{sec:creazione_db}).
Per eseguire il server recarsi nella cartella \textit{TheKnife/bin} 
e lanciare il comando:
\begin{minted}{bash}
java -jar Server.jar
\end{minted}
Una volta avviato, il server richiederà le credenziali di accesso al database.
Le credenziali sono:
\begin{itemize}
    \item \textbf{Username:} \texttt{theknife}
    \item \textbf{Password:} \texttt{password}
\end{itemize}

\subsection{Avvio del client}
\label{sec:avvio_client}
Per avviare il client è necessario che il server sia in esecuzione
e raggiungibile sulla porta \texttt{1099}(\ref{sec:avvio_server}) dal dispositivo che 
esegue il client.
Per eseguire il client recarsi nella cartella \textit{TheKnife/bin} 
e lanciare il comando:
\begin{minted}{bash}
java --module-path ~/path/to/javafx-sdk-17.0.15/lib \
--add-modules javafx.controls,javafx.fxml -jar ClientTK.jar
\end{minted}
Sostituire \texttt{\~/path/to/javafx-sdk-17.0.15/lib} con 
il percorso dove è presente la propria libreria JavaFX.
Una volta avviato apparirà la finestra principale con interfaccia grafica
e sarà possibile effettuare il login con le credenziali di un utente, 
registrarsi o accedere come utente ospite.
