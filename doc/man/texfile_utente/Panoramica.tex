\section{Panoramica del Sistema}
\label{cap:panoramica}
\subsection{Architettura generale}
\emph{TheKnife} adotta un'architettura client–server modulare:
\begin{itemize}
    \item \textbf{Server backend:} implementato in Java 24, espone servizi Registry RMI e gestisce la logica di persistenza su PostgreSQL.
    \item \textbf{Client desktop:} applicazione JavaFX che interagisce con il server tramite JavaRMI.
    \item \textbf{Database:} PostgreSQL 13+ come DBMS
\end{itemize}

\subsection{Tecnologie e librerie}
\begin{description}
    \item[PostgreSQL 13+] Sistema di gestione del database open source.
    \item[JavaFX] Libreria per l'interfaccia grafica (client).
    \item[argon2] Algoritmo di hashing per la sicurezza delle password.
\end{description}
