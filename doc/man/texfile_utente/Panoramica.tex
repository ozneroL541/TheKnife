\section{Panoramica del Sistema}
\label{cap:panoramica}
\subsection{Architettura generale}
\emph{TheKnife} adotta un'architettura client-server modulare:
\begin{itemize}
    \item \textbf{Server - \texttt{Server.jar}} implementato in Java 24, 
    espone servizi Registry RMI sulla porta 1099 e 
    gestisce i dati interfacciandosi con un database relazionale.
    \item \textbf{Client desktop - \texttt{ClientTK.jar}} applicazione con 
    Graphical User Interface implementata in JavaFX 
    che interagisce con il server tramite JavaRMI.
    \item \textbf{Database} database creato su PostgreSQL 13+
\end{itemize}

\subsection{Tecnologie e librerie}
Nel seguente elenco sono riportate le principali teconologie del 
progetto, necessarie per la sua compilazione ed esecuzione.
\begin{description}
    \item[Java Virtual Machine (JVM) 22+] Sistema di esecuzione per applicazioni Java.
    \item[Maven 3.9.9] Strumento per la compilazione e gestione delle dipendenze.
    \item[PostgreSQL 13+] Sistema di gestione del database open source.
    \item[JavaFX] Libreria per l'interfaccia grafica (client).
\end{description}
