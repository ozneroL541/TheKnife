\section{Sviluppo}
\subsection{Architettura}
Il sistema utilizza un'architettura client-server basata su 
RMI (Remote Method Invocation) per la comunicazione tra client 
e server.
Il server si interfaccia con un database PostgreSQL tramite JDBC,
gestendo la logica di persistenza tramite DAO (Data Access Object).
Il client, sviluppato in JavaFX, implementa il pattern MVC (Model-View-Controller)
per la gestione dell'interfaccia utente e delle interazioni.
\subsection{Database}
\subsubsection{Configurazione e connessione}
\label{sec:db-connessione}
Il sistema utilizza PostgreSQL 13+ con il driver JDBC.
Durante la compilazione è richiesto che l'utente \textit{postgre},
con password \textit{postgre}, abbia i permessi per creare un 
nuovo database e corrispetttivo utente \textit{theknife}.\\
Il database creato avrà i seguenti parametri di connessione:
\begin{itemize}
    \item \textbf{Database} \texttt{\detokenize{theknife_db}}
    \item \textbf{User:} \texttt{theknife}
    \item \textbf{Password:} \texttt{password}
    \item \textbf{Host:} \texttt{localhost}
    \item \textbf{Port:} \texttt{5432}
\end{itemize}

\subsubsection{Struttura delle tabelle}
\begin{description}
Di seguito riportata la struttura delle tabelle del database,
con le chiavi primarie sottolineate e le chiavi esterne in notazione \texttt{tabella\textsuperscript{riferimento}}.
Le tabelle sono state create in modo da rispettare le relazioni definite nel diagramma ER.\\
Per la creazione delle tabelle e il popolamento iniziale del database sono stati utilizzati script SQL.
\begin{itemize}
    \item \textbf{addresses} (\underline{address\_id}, country, city, street, house\_number, latitude, longitude)
    \item \textbf{users} (\underline{username}, h\_password, name, surname, birth\_date, role, address\_id\textsuperscript{addresses})
    \item \textbf{restaurants} (\underline{restaurant\_id}, r\_owner, r\_name, avg\_price, delivery, booking, r\_type, address\_id\textsuperscript{addresses})
    \item \textbf{favorites} (\underline{username\textsuperscript{users}, restaurant\_id\textsuperscript{restaurants}})
    \item \textbf{reviews} (\underline{username\textsuperscript{users}, restaurant\_id\textsuperscript{restaurants}}, rating, comment, reply)
\end{itemize}
\end{description}
Le password di tutti gli utenti sono cifrate con Argon2, 
un algoritmo di hashing sicuro resistente agli attacchi di 
forza bruta, anche parallelizzati.

\subsubsection{Inizializzazione}
L'uso di \texttt{sql-maven-plugin} permette di gestire le seguenti funzioni.
\begin{enumerate}
  \item Creazione database e utente (\texttt{create\_database.sql}).
  \item Creazione tabelle (script: \texttt{addresses.sql}, \dots, \texttt{reviews.sql}).
  \item Popolamento con dati di esempio (\texttt{samples/}).
\end{enumerate}

\subsection{Data Access Object - DAO}
\label{sec:dao}
Si è adottato il pattern DAO per isolare la logica di persistenza.\\
Le funzionalità garantite da tale pattern sono:
\begin{itemize}
  \item \textbf{UserDAO}: autenticazione, registrazione, gestione profilo.
  \item \textbf{RestaurantDAO}: ricerca avanzata, preferiti, gestione ristoranti.
  \item \textbf{ReviewDAO}: creazione/modifica/cancellazione recensioni e risposte.
\end{itemize}

\paragraph{UserDAO}
\begin{itemize}
    \item \textbf{Autenticazione}: Verifica delle credenziali durante il processo di login, con controllo della password cifrata
    \item \textbf{Registrazione}: Inserimento di nuovi utenti nel sistema con validazione dei dati e cifratura della password
    \item \textbf{Gestione profili}: Operazioni di lettura e aggiornamento delle informazioni utente
    \item \textbf{Controllo ruoli}: Distinzione tra utenti clienti e ristoratori per l'autorizzazione delle operazioni
\end{itemize}

\paragraph{RestaurantDAO}
\begin{itemize}
    \item \textbf{Ricerca parametrica}: Implementazione della funzione \texttt{searchRestaurant()} con filtri multipli per:
    \begin{itemize}
        \item Tipologia di cucina
        \item Localizzazione geografica (parametro obbligatorio)
        \item Fascia di prezzo
        \item Disponibilità servizio delivery
        \item Disponibilità prenotazione online
        \item Rating minimo
    \end{itemize}
    \item \textbf{Gestione ristoranti}: Creazione dei ristoranti
    \item \textbf{Calcolo metriche}: Aggregazione di dati per rating medio e numero recensioni
\end{itemize}

\paragraph{ReviewDAO}
\begin{itemize}
    \item \textbf{Gestione recensioni}: Operazioni CRUD (Create, Read, Update, Delete) per le recensioni
    \item \textbf{Risposte ristoratori}: Implementazione di \texttt{replyToReview()} con controllo autorizzazioni
    \item \textbf{Visualizzazione}: Recupero recensioni per ristorante con \texttt{getReviews()}
    \item \textbf{Aggregazioni}: Calcolo rating medio e conteggio recensioni per ristorante
\end{itemize}

\paragraph{FavoritesDAO}
\begin{itemize}
    \item \textbf{Aggiunta preferiti}: Implementazione di \texttt{addFavorite()}
    \item \textbf{Rimozione preferiti}: Implementazione di \texttt{removeFavorite()}
    \item \textbf{Visualizzazione}: Recupero lista ristoranti preferiti con \texttt{getFavorites()}
\end{itemize}

\paragraph{Prepared Statements}
Tutte le queries utilizzano PreparedStatement per prevenire SQL injection e migliorare le performance attraverso il caching delle queries compilate.

\subsection{Architettura RMI}
\label{sec:rmi}
Il sistema TheKnife implementa un'architettura distribuita client-server basata su Java
RMI (Remote Method Invocation) per gestire la comunicazione tra i client e il server.
Questa scelta architetturale permette di separare la logica di presentazione (client) dalla
logica di business e accesso ai dati (server), garantendo scalabilit`a e manutenibilit`a del
sistema.
\subsection{Struttura}
\paragraph{Registry RMI}
Il server TheKnife utilizza il registry RMI per registrare i servizi disponibili ai client. Il registry viene avviato sulla porta standard 1099 e permette ai client di ottenere riferimenti remoti ai servizi tramite lookup per nome.

\paragraph{Interfacce Remote}
Tutte le interfacce di servizio estendono \texttt{java.rmi.Remote} e dichiarano che i metodi possono lanciare \texttt{RemoteException}. Questa struttura garantisce la trasparenza della distribuzione per i client che invocano metodi remoti.

\subsection{Servizi implementati}
\begin{description}
  \item[\texttt{UserService}] Autenticazione, registrazione, gestione profilo.
  \item[\texttt{RestaurantService}] Ricerca avanzata, preferiti, gestione ristoranti.
  \item[\texttt{ReviewService}] Creazione/modifica/cancellazione recensioni e risposte.
\end{description}

\subsection{Data Transfer Objects - DTO}
Tutti i DTO implementano \texttt{Serializable} per permettere il trasferimento attraverso RMI. Gli oggetti vengono serializzati automaticamente dal meccanismo RMI durante il trasporto tra client e server.
\paragraph{UserDTO}
Incapsula i dati dell'utente con campi pubblici per accesso diretto:
\begin{itemize}
    \item Informazioni di base: \texttt{username}, \texttt{name}, \texttt{surname}
    \item Credenziali: \texttt{password} (solo per operazioni di autenticazione)
    \item Localizzazione: \texttt{address}, \texttt{latitude}, \texttt{longitude}
    \item Metadati: \texttt{birth\_date}, \texttt{role} (cliente/ristoratore)
\end{itemize}

\paragraph{RestaurantDTO}
Contiene tutte le informazioni del ristorante secondo le specifiche:
\begin{itemize}
    \item Identificazione: \texttt{restaurant\_id}, \texttt{r\_name}
    \item Localizzazione: \texttt{address}, \texttt{latitude}, \texttt{longitude}
    \item Caratteristiche: \texttt{avg\_price}, \texttt{avg\_rating}, \texttt{r\_type} (tipologia cucina)
    \item Servizi: \texttt{delivery}, \texttt{booking}
    \item Recensioni: \texttt{ArrayList<ReviewDTO> reviews}
\end{itemize}

\paragraph{SearchCriteriaDTO}
Implementa il pattern Builder per costruire criteri di ricerca flessibili:
\begin{verbatim}
SearchCriteriaDTO criteria = SearchCriteriaDTO.builder()
    .coordinates(45.8983, 8.8289)
    .cuisineType(CuisineType.ITALIAN)
    .priceRange(215.0, 40.0)
    .deliveryAvailable(true)
    .minRating(3.0)
    .build();
\end{verbatim}

\subsection{Architettura del Client}
\label{sec:client}
Il modulo client di TheKnife implementa un'architettura basata 
sul pattern Model-View-Controller (MVC) utilizzando JavaFX come 
framework per l'interfaccia grafica. Questa scelta architetturale 
garantisce una separazione netta tra la logica di presentazione, 
la gestione degli eventi e la comunicazione con i servizi remoti.

\subsection{Tecnologie e Framework Utilizzati}

\paragraph{JavaFX e FXML}
L'interfaccia utente è sviluppata utilizzando JavaFX con file FXML per la definizione dichiarativa delle viste. Questa approccio offre diversi vantaggi:

\begin{itemize}
    \item \textbf{Separazione design-logica}: I file FXML permettono di definire la struttura e l'aspetto dell'interfaccia separatamente dalla logica applicativa
    \item \textbf{Manutenibilità}: Le modifiche all'interfaccia possono essere effettuate nei file FXML senza toccare il codice Java
    \item \textbf{Designer-friendly}: I file FXML possono essere creati e modificati con tool grafici come Scene Builder
    \item \textbf{Riusabilità}: Componenti UI possono essere facilmente riutilizzati tra diverse viste
\end{itemize}

Ogni vista dell'applicazione è definita in un file FXML separato (es. \texttt{login-view.fxml}, \texttt{search-view.fxml}) che specifica la struttura dei componenti, il layout e le proprietà di styling.

\paragraph{Pattern MVC nel Client}
L'architettura client implementa il pattern MVC dove:
\begin{itemize}
    \item \textbf{Model}: Rappresentato dai DTO ricevuti dai servizi RMI
    \item \textbf{View}: Definita nei file FXML con componenti JavaFX
    \item \textbf{Controller}: Classi Java che gestiscono la logica di presentazione e l'interazione utente
\end{itemize}

\subsection{Gestione sessione utente}
Implementata con un Singleton (\texttt{UserSession}) che mantiene stato di login, dati utente e controlli di autorizzazione.
