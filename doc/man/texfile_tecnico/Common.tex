\section{Common}
Il package \texttt{Common} contiene i files comuni al 
client e al server per permetterne l'interazione.
\`E il package principale del progetto su cui si basa 
l'intera architettura del sistema applicativo.

\subsection{Data Transfer Objects}
Il design pattern \textit{Data Transfer Object} (DTO) è 
utilizzato per trasferire dati tra il client e il server. 
I DTO sono classi che contengono solo i dati necessari per la 
comunicazione, senza logica di business. 
Questo approccio semplifica la serializzazione e deserializzazione 
dei dati durante le richieste poiché tutte le classi DTO 
implementano \texttt{Serializable}.\\
Le classi principali che utilizzano il design pattern corrispondono 
alle entità principali del database e sono le seguenti:
\begin{itemize}
    \item \texttt{\hyperref[sec:addressdto]{AddressDTO}}
    \item \texttt{\hyperref[sec:userdto]{UserDTO}}
    \item \texttt{\hyperref[sec:restaurantdto]{RestaurantDTO}}
    \item \texttt{\hyperref[sec:reviewdto]{ReviewDTO}}
\end{itemize}
Come si può notare dalla lista, non è stato creato un DTO per
l'entità \textit{Favorite} poiché non si è ritenuto necessario 
creare una classe apposita per la sua gestione. 
\`E stata invece utilizzata nei metodi interni al server per le 
elaborazioni, ma non come oggetto trasferibile.\\
Si è ritenuto opportuno creare due tipi enumerabili
per rappresentare i tipi di cucina e di utenti:
\begin{itemize}
    \item \texttt{\hyperref[sec:cuisinetype]{CuisineType}}
    \item \texttt{\hyperref[sec:userroledto]{UserRoleDTO}}
\end{itemize}
Per facilitare la gestione della ricerca e i relativi filtri è 
stata creata una classe apposita chiamata 
\texttt{\hyperref[sec:searchcriteriadto]{SearchCriteriaDTO}}.

\subsubsection{AddressDTO}
\label{sec:addressdto}
La classe \texttt{AddressDTO} rappresenta un indirizzo
e contiene campi analoghi a quelli della tabella
\textit{addresses} del database ad eccezione del campo
\textit{address\_id}.

\subsubsection{UserDTO}
\label{sec:userdto}
La classe \texttt{UserDTO} rappresenta un utente
e contiene campi analoghi a quelli della tabella
\textit{users}.\\
L'indirizzo dell'utente è rappresentato da un campo
\textit{address} di tipo \texttt{AddressDTO} che durante 
l'inserimento nel database viene convertito nell'ID corrispondente 
e in fase di lettura nel corrispettivo oggetto.\\
L'attributo \textit{role} è di tipo \texttt{UserRoleDTO} ed è 
gestito dalla classe enumerabile \texttt{UserRoleDTO}.

\subsubsection{RestaurantDTO}
\label{sec:restaurantdto}
La classe \texttt{RestaurantDTO} rappresenta un ristorante
e contiene campi analoghi a quelli della tabella
\textit{restaurants}.\\
L'indirizzo dell'utente è rappresentato da un campo
\texttt{address} di tipo \texttt{AddressDTO} che durante 
l'inserimento nel database viene convertito nell'ID corrispondente 
e in fase di lettura nel corrispettivo oggetto.\\
Sono presenti due campi aggiuntivi per fornire maggiori 
informazioni. I campi sono: \texttt{rating} e 
\texttt{reviewsNumber}. Essendo essi deducibili dal Database è 
necessario poterli trasferire al client e si è ritenuto opportuno 
farlo inserendoli nell'oggetto dalla cui entita corrispettiva si 
derivano.\\

\subsubsection{ReviewDTO}
\label{sec:reviewdto}
La classe \texttt{ReviewDTO} rappresenta una recensione ad un 
ristorante e contiene campi analoghi a quelli della tabella
\textit{reviews}.\\
L'utente che ha scritto la recensione è rappresentato dal campo 
\texttt{username} di tipo \texttt{String} e il ristorante dal campo 
\texttt{restaurant\_id} di tipo \texttt{String}.
Non è stato ritenuto opportuno utilizzare dei tipo semplici piuttosto
che i rispettivi DTO poiché si ritiene che essi siano sufficienti e 
che invece inserire ulteriori oggetti possa causare inutile overhead.\\

\subsubsection{CuisineType}
\label{sec:cuisinetype}
La classe \texttt{CuisineType} è un tipo enumerabile che
rappresenta i 276 tipi di cucina disponibili nel sistema.
I tipi di cucina sono predefiniti e sono gli medesimi sia per 
la classe \texttt{CuisineType} che per il database.
Questo approccio consente di evitare errori di battitura e
di garantire la coerenza tra applicazione e database.\\
I tipi di cucina sono rappresentati come stringhe e sono
utilizzati per filtrare i ristoranti durante le ricerche.

\subsubsection{UserRoleDTO}
\label{sec:userroledto}
La classe \texttt{UserRoleDTO} è un tipo enumerabile che
rappresenta i ruoli degli utenti nel sistema.
I ruoli sono:
\begin{itemize}
    \item \texttt{CLIENT}: cliente registrato
    \item \texttt{OWNER}: proprietario di un ristorante
\end{itemize}
I ruoli sono predefiniti e identici sia per l'applicazione 
Java che per il database.

\subsubsection{SearchCriteriaDTO}
\label{sec:searchcriteriadto}
La classe \texttt{SearchCriteriaDTO} rappresenta i criteri di 
ricerca per i ristoranti.
I suoi campi rappresentano i filtri applicabili alla ricerca.
\begin{itemize}
    \item \texttt{cuisineType}: tipologia di cucina
    \item \texttt{minPrice}: prezzo minimo in euro
    \item \texttt{maxPrice}: prezzo massimo in euro
    \item \texttt{deliveryAvailable}: disponibilità del servizio di consegna
    \item \texttt{onlineBookingAvailable}: disponibilità della prenotazione
    \item \texttt{minRating}: valutazione minima (scala 1-5)
    \item \texttt{latitude}: latitudine dell'utente
    \item \texttt{longitude}: longitudine dell'utente
\end{itemize}
Tutti i campi sono opzionali ad eccezione delle coordinate, 
necessarie a ordinare i ristoranti in base alla distanza
dall'utente tramite la 
\textit{\href{https://rosettacode.org/wiki/Haversine_formula}{Formula di Haversine}}.
Per la creazione dell'oggetto \texttt{SearchCriteriaDTO} è stato
utilizzato il design pattern \textit{Builder} per facilitarne la
creazione e la gestione dei criteri.
Il pattern \textit{Builder} consente di creare oggetti complessi
in modo flessibile e leggibile, permettendo di impostare solo i
campi desiderati senza dover gestire costruttori con molti 
parametri.
Questo approccio permette flessibilità senza compromettere la
leggibilità del codice, facilitando la creazione dell'oggetto
\texttt{SearchCriteriaDTO}.
