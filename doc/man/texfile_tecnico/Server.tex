\section{Server}
Il package \texttt{Server} contiene le classi necessarie per 
la gestione del server, che si occupa di gestire le richieste 
dai client interfacciandosi con il database.

\subsection{Main}
La classe \texttt{ServerTK} contiene il metodo \texttt{main} 
che avvia il server e gestisce le connessioni dei client.
Essa delega la richiesta di credenziali e la connessione al 
database alla classe \texttt{DBConnector}.
La classe \texttt{DBConnection} invece è la vera e propria 
istanza di connessione al database implementata con il design
pattern \textit{Singleton}, garantendo che ci sia una sola connessione
attiva alla volta.

\subsection{Data Access Object}
Il package \texttt{dao} contiene le classi che implementano 
il pattern Data Access Object (DAO) per l'interazione
con il database. Queste classi sono responsabili dell'interazione 
del database con l'applicazione.
Le classi implementate sono le seguenti:
\begin{itemize}
    \item \texttt{UserDAO}: gestisce le operazioni relative agli utenti
    \item \texttt{RestaurantDAO}: gestisce le operazioni relative ai ristoranti
    \item \texttt{AddressDAO}: gestisce le operazioni relative agli indirizzi
    \item \texttt{ReviewDAO}: gestisce le operazioni relative alle recensioni
\end{itemize}
Queste classi semplificano la gestione dei dati permettendo 
di interfacciarsi al database senza preoccuparsi delle specifiche
implementative dello stesso. 
In queste classi sono implementate le query per l'inserimento, 
la modifica e l'eliminazione dei dati e gestiscono in autonomia 
le eccezioni che possono verificarsi durante l'interazione
con il database.

\subsection{Server Services}
Il package \texttt{server\_services} contiene le classi che implementano
le interfacce definite nel package \texttt{common}.
Queste classi vengono esposte dal database in un \texttt{Registry}
e rese accessibili ai client tramite JavaRMI.
I metodi delle classi garantiscono una corretta gestione della 
concorrenza e utilizzano le classi DAO per interagire con il database.
Le classi implementate sono le seguenti:
\begin{itemize}
    \item \texttt{RestaurantServiceImpl.java} implementa \texttt{RestaurantService}
    \item \texttt{ReviewServiceImpl.java} implementa \texttt{ReviewService}
    \item \texttt{UserServiceImpl.java} implementa \texttt{UserService}
\end{itemize}
