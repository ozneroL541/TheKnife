\section{Client}
Il package \texttt{Client} contiene le classi necessarie per
la gestione del client, che si occupa di interagire con l'utente 
attraverso la Graphical User Interface e di utilizzare i 
servizi offerti dal server tramite JavaRMI.

\subsection{ClientTK}
La classe \texttt{ClientTK} contiene il metodo \texttt{main}
che all'avvio lancia l'interfaccia grafica dell'applicazione e 
istaura una connession col server per poter utilizzare i servizi 
offerti tramite il Registro RMI.\\
La connessione con il Registro è unica per ogni client e 
viene effettuata tramite la classe \texttt{ServerAddress} che, 
oltre a contenere le informazioni riguardanti l'indirizzo IP (o 
corrispettivo hostname) e la porta del server, contiene anche il metodo
\texttt{getRegistry()} (unico metodo pubblico fornito dalla classe) 
che restituisce l'istanza \texttt{Registry} del server.
La classe \texttt{ServerAddress} è implementata con il design pattern 
\textit{Singleton} per garantire che ogni client ne utilizzi una sola 
istanza così da evitare che vi siano più connessioni da uno stesso +
client e ridurre così il carico sul server.

\subsection{Sessione Utente}
La classe \texttt{UserSession} contiene le informazioni relative
allo stato della sessione.
Essa è implementata con il design pattern \textit{Singleton} 
per garantirne l'unicità.
I campi principali sono \texttt{isLoggedIn} e \texttt{currentUser}.
Il primo è un campo booleano che indica se l'utente è 
attualmente loggato, il secondo è un oggetto \texttt{UserDTO} che 
contiene le informazioni dell'utente loggato.
Oltre a conoscere lo stato di log di un utente è possibile, 
qualora l'utente sia loggato, saperne la tipologia così da 
poter differenziare l'esperienza tra cliente e ristoratore.
Il campo \texttt{currentUser} è determinante anche per la ricerca
dei ristoranti in quanto, contenendo le coordinate dell'utente, 
permette di trovare i ristoranti in base alla loro distanza.

\subsection{User Interface}