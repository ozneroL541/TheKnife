\section{Client}
Il package \texttt{Client} contiene le classi necessarie per
la gestione del client, che si occupa di interagire con l'utente 
attraverso la Graphical User Interface e di utilizzare i 
servizi offerti dal server tramite JavaRMI.
L'architettura client implementa il pattern MVC dove:
\begin{itemize}
    \item \textbf{Model}: Rappresentato dai DTO ricevuti dai servizi RMI
    \item \textbf{View}: Definita nei file FXML con componenti JavaFX
    \item \textbf{Controller}: Classi Java che gestiscono la logica di presentazione e l'interazione utente
\end{itemize}

\subsection{ClientTK}
La classe \texttt{ClientTK} contiene il metodo \texttt{main}
che all'avvio lancia l'interfaccia grafica dell'applicazione e 
istaura una connession col server per poter utilizzare i servizi 
offerti tramite il Registro RMI.\\
La connessione con il Registro è unica per ogni client e 
viene effettuata tramite la classe \texttt{ServerAddress} che, 
oltre a contenere le informazioni riguardanti l'indirizzo IP (o 
corrispettivo hostname) e la porta del server, contiene anche il metodo
\texttt{getRegistry()} (unico metodo pubblico fornito dalla classe) 
che restituisce l'istanza \texttt{Registry} del server.
La classe \texttt{ServerAddress} è implementata con il design pattern 
\textit{Singleton} per garantire che ogni client ne utilizzi una sola 
istanza così da evitare che vi siano più connessioni da uno stesso +
client e ridurre così il carico sul server.

\subsection{Sessione Utente}
La classe \texttt{UserSession} contiene le informazioni relative
allo stato della sessione.
Essa è implementata con il design pattern \textit{Singleton} 
per garantirne l'unicità.
I campi principali sono \texttt{isLoggedIn} e \texttt{currentUser}.
Il primo è un campo booleano che indica se l'utente è 
attualmente loggato, il secondo è un oggetto \texttt{UserDTO} che 
contiene le informazioni dell'utente loggato.
Oltre a conoscere lo stato di log di un utente è possibile, 
qualora l'utente sia loggato, saperne la tipologia così da 
poter differenziare l'esperienza tra cliente e ristoratore.
Il campo \texttt{currentUser} è determinante anche per la ricerca
dei ristoranti in quanto, contenendo le coordinate dell'utente, 
permette di trovare i ristoranti in base alla loro distanza.

\subsection{User Interface}
L'interfaccia grafica è stata interamente realizzata tramite 
il framework JavaFX.
Essa è divisa in più viste, ognuna delle quali è definita 
tramite un file \textit{FXML} presente nella cartella 
\textit{TheKnife/client/src/main/resources/it/uninsubria/controller/}.
Ogni componente grafico è nominato come \textit{name-view.fxml} ed 
è corrisposta da una classe Java nominata \textit{NameController.java}
che ne gestisce gli eventi e le interazioni con l'utente.\\
I componenti importanti sono:
\begin{itemize}
    \item  \textbf{AddReply} finestra per l'aggiunta di una risposta ad una recensione (visibile solo dai ristoratori)
    \item  \textbf{AddRestaurant} fiinestra per l'aggiunta di un ristorante (visibile solo dai ristoratori)
    \item  \textbf{AddReview} finestra per l'aggiunta di una recensione (visibile solo dagli utenti registrati)
    \item  \textbf{Login} finestra per il login
    \item  \textbf{MyArea} finestra per la gestione dell'area personale dell'utente (visibile solo agli utenti registrati)
    \item  \textbf{Registration} finestra per la registrazione di un nuovo utente
    \item  \textbf{RestaurantInfo} finestra per la visualizzazione delle informazioni relative ad un ristorante
    \item  \textbf{Search} finestra per la ricerca dei ristoranti
\end{itemize}

\subsubsection{Components}
Nella cartella 
\textit{TheKnife/client/src/main/java/it/uninsubria/controller/ui\_components/}
sono presenti dei componenti grafici riutilizzabili in più finestre.\\
Essi sono:
\begin{itemize}
    \item \textbf{GenericResultsComponent} componente generico (ristorante o recensione) per mostrare una lista di \textit{cards} in modo che si possa scorrere
    \item \textbf{RestaurantCardComponent} componente per mostrare i ristoranti come \textit{cards} clickabili
    \item \textbf{ReviewCardComponent} componente per mostrare le recensioni come \textit{cards} clickabili
\end{itemize}

\subsection{Util}
Il package \texttt{utilclient} contiene la classe \texttt{UtilClient} che
che fornisce un metodo per il calcolo della distanza tra due coordinate
geografiche. 
Questo metodo è utilizzato in vari punti del client.

\begin{minted}{java}
public static double calculateDistance(double lat1, double lon1, double lat2, double lon2) {
    /** Radius of the Earth in kilometers */
    final int EARTH_RADIUS = 6371;
    /** Latitude distances in radians */
    double latDistance = Math.toRadians(lat2 - lat1);
    /** Longitude distances in radians */
    double lonDistance = Math.toRadians(lon2 - lon1);
    /* Haversine Formula */
    double a = Math.sin(latDistance / 2) * Math.sin(latDistance / 2)
            + Math.cos(Math.toRadians(lat1)) * Math.cos(Math.toRadians(lat2))
            * Math.sin(lonDistance / 2) * Math.sin(lonDistance / 2);
    double c = 2 * Math.atan2(Math.sqrt(a), Math.sqrt(1 - a));
    return EARTH_RADIUS * c;
}
\end{minted}
